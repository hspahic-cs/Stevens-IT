%
%    Spin Reference Card
%
%  Copyright 2007 by Mordechai (Moti) Ben-Ari.
%  This work is licensed under the Creative Commons
%  Attribution-Noncommercial-ShareAlike 3.0 License. 
%  To view a copy of this license, 
%  visit http://creativecommons.org/licenses/by-nc-sa/3.0/; 
%  or, (b) send a letter to Creative Commons, 
%  543 Howard Street, 5th Floor, San Francisco, California, 94105, USA.
%
\documentclass{leaflet}
%\documentclass[notumble,nocombine]{leaflet}  % One page per column
\usepackage{mathptmx}
\usepackage{url}
\newcommand{\sct}[1]{\subsubsection{#1}\mbox{}\\}
\newcommand{\spc}{\hspace*{1.5em}}
\CutLine*{1}\CutLine*{3}\CutLine*{4}\CutLine*{6}

\title{\vspace*{-2ex}Spin Reference Card}
\author{\vspace*{-2ex}Mordechai (Moti) Ben-Ari\vspace*{-2ex}}

\begin{document}
\maketitle
\thispagestyle{empty}
\vspace*{-2ex}
{\scriptsize Copyright 2007 by Mordechai (Moti) Ben-Ari.
This work is licensed under the Creative Commons
Attribution-Noncommercial-ShareAlike 3.0 License. 
To view a copy of this license, 
visit \url{http://creativecommons.org/licenses/by-nc-sa/3.0/}; 
or, (b) send a letter to Creative Commons, 
543 Howard Street, 5th Floor, San Francisco, California, 94105, USA.}

\sct{Datatypes}
\texttt{bit} (1 bit)\\
\texttt{bool} (1 bit)\\
\texttt{byte} (8 bits unsigned)\\
\texttt{short} (16$^{*}$ bits signed)\\
\texttt{int} (32$^{*}$ bits signed)\\
\texttt{unsigned} ($\leq$ 32$^{*}$ bits unsigned)\\
\spc{}$^{*}$ - for a 32-bit machine.\\
\texttt{pid}\\
\texttt{chan}\\
\texttt{mtype} = \{ name, name, ... \} (8 bits) \\
\texttt{typedef} typename \{ sequence of declarations \}\\
\\
Declaration - type var [= initial value]\\
Default initial values are zero.\\
Array declaration - type var[N] [= initial value]\\
Array initial value assigned to all elements.

\sct{Operators (descending precedence)}
\spc{}\verb+()+ \spc{}\verb+[]+ \spc{}\verb+.+ \\
\spc{}\verb+!+ \spc{}\verb+~+ \spc{}\verb=++= \spc{}\verb+--+\\
\spc{}\verb+*+ \spc{}\verb+/+ \spc{}\verb+%+\\
\spc{}\verb=+= \spc{}\verb+-+ \\
\spc{}\verb+<<+ \spc{}\verb+>>+ \\
\spc{}\verb+<+ \spc{}\verb+<=+ \spc{}\verb+>+ \spc{}\verb+>= +\\
\spc{}\verb+==+ \spc{}\verb+!=+\\
\spc{}\verb+&+ \\
\spc{}\verb+^+ \\
\spc{}\verb=|= \\
\spc{}\verb+&&+\\
\spc{}\verb=||=\\
\spc{}\verb+( ... -> ... : ... )+ conditional expression\\
\spc{}\verb+=+

\sct{Predefined}
Constants - \texttt{true}, \texttt{false}\\
Variables (read-only except \texttt{\_}):\\
\spc{}\texttt{\_} - write-only hidden scratch variable\\
\spc{}\texttt{\_nr\_pr} - number of processes\\
\spc{}\texttt{\_pid} - instantiation number of executing process\\
\spc{}\texttt{timeout} - no executable statements in the system?

\sct{Preprocessor}
\texttt{\#define} name (arguments) string\\
\texttt{\#undef}, \texttt{\#if}, \texttt{\#ifdef}, \texttt{\#ifndef}, \texttt{\#else}, \texttt{\#endif}\\
\texttt{\#include} "file name"\\
\texttt{inline} name (arguments) \{ ... \}

\sct{Statements} 
Assignment - var = expression, var\verb=++=, var\verb=--=\\
\texttt{assert}(expression)\\
\\
\texttt{printf}, \texttt{printm} - print to standard output\\
\spc{}\verb=%c= (character), \verb=%d= (decimal), \verb=%e= (\texttt{mtype}),\\
\spc{}\verb=%o= (octal), \verb=%u= (unsigned), \verb=%x= (hex)\\
\texttt{scanf} - read from standard input in simulation mode\\
\\
\texttt{skip} - no operation\\
\texttt{break} - exit from innermost \texttt{do} loop\\
\texttt{goto} - jump to label\\
Label prefixes with a special meaning:\\
\spc{}\texttt{accept} - accept cycle\\
\spc{}\texttt{end} - valid end state\\
\spc{}\texttt{progress} - non-progress cycle\\
\\
\texttt{atomic} \{ ... \} - execute without interleaving\\
\texttt{d\_step} \{ ... \} - execute deterministically (no jumping 
in or out; deterministic choice among true guards; only the first
statement can block).\\
\\
\{ ... \} \texttt{unless} \{ ... \} - exception handling.

\sct{Guarded commands}
\texttt{if} :: guard \verb=->= statements :: ... \texttt{fi}\\
\texttt{do} :: guard \verb=->= statements :: ... \texttt{od}\\
\texttt{else} guard - executed if all others are false.

\sct{Processes}
Declaration - \texttt{proctype} procname (parameters) \{ ... \}\\
Activate with prefixes - \texttt{active} or \texttt{active}[N]\\
Explicit process activation - \texttt{run} procname (arguments)\\
Initial process - \texttt{init} \{ ... \}\\
Declaration suffixes:\\
\spc{}\texttt{priority} - set simulation priority\\
\spc{}\texttt{provided} (e) - executable only if expression e is true

\sct{Channels}
\texttt{chan} ch = [ capacity ] \texttt{of} \{ type, type, ... \}\\
\\
\begin{tabular}{@{\hspace*{0pt}}ll}
ch ! args & send\\
ch !! args & sorted send\\
\\
ch ? args & receive and remove if \emph{first} message matches\\ 
ch ?? args & receive and remove if \emph{any} message matches\\ 
ch ? \verb+<+args\verb+>+ & receive if \emph{first} message matches\\ 
ch ?? \verb+<+args\verb+>+ & receive if \emph{any} message matches\\ 
ch ? [args] & poll \emph{first} message (side-effect free)\\ 
ch ?? [args] & poll \emph{any} message (side-effect free)\\
\\
\end{tabular}
Matching in a receive statement: constants and \texttt{mtype} symbols must match;
variables are assigned the values in the message; 
\texttt{eval}(expression) forces a match with the current
value of the expression.\\
\\
\texttt{len}(ch) - number of messages in a channel\\
\texttt{empty}(ch) / \texttt{nempty}(ch) - is channel empty / not empty?\\
\texttt{full}(ch) / \texttt{nfull}(ch) - is channel full / not full?\\
\\
Channel use assertions:\\
\spc{}\texttt{xr} ch - channel ch is receive-only in this process\\
\spc{}\texttt{xs} ch - channel ch is send-only in this process

\newpage

\sct{Temporal logic}

\vspace*{-2ex}\spc{}
\begin{tabular}{cl}
\verb+!+ & not\\
\verb+&&+ & and\\
\verb+||+ & or\\
\verb+->+ & implies\\
\verb+<->+ & equivalent to\\
&\\
\verb+[]+ & always\\
\verb+<>+ & eventually\\
\verb+X+ & next\\
\verb+U+ & strong until\\
\verb+V+ & dual of U defined as \verb+pVq <-> !(!pU!q)+
\end{tabular}

\sct{Remote references}
Test the control state or the value of a variable:\\
\spc{}process-name \verb+@+ label-name\\
\spc{}proctype-name [ expression ] \verb+@+ label-name\\
\spc{}process-name  : label-name\\
\spc{}proctype-name [ expression ] : label-name

\sct{Never claim}
\texttt{never} \{ ... \}.\\
Predefined constructs that can only appear in a never claim:\\
\spc{}\texttt{\_last} - last process to execute\\
\spc{}\texttt{enabled}(p) - is process enabled?\\
\spc{}\texttt{np\_} - true if no process is at a progress label\\
\spc{}\texttt{pc\_value}(p) - current control state of process\\
\spc{}remote references\\
See also \texttt{trace} and \texttt{notrace}.

\sct{Variable declaration prefixes}
\texttt{hidden} - hide this variable from the system state\\
\texttt{local} - a global variable is accessed only by one process\\
\texttt{show} - track variable in Xspin message sequence charts

\sct{Verification}
Safety:\\
\spc{}\verb+spin -a file+\\
\spc{}\verb+gcc -DSAFETY -o pan pan.c+\\
\spc{}\verb+pan+ or \verb+./pan+\\
\spc{}\verb+spin -t -p -l -g -r -s file+

\newpage

Liveness:\\
\spc{}\verb+spin -a file+\\
\spc{}\verb+gcc -o pan pan.c+\\
\spc{}\verb+pan -a -f+ or \verb+./pan -a -f+\\
\spc{}\verb+spin -t -p -l -g -r -s file+

\sct{Spin arguments}
\begin{tabular}{ll}
\texttt{-a} & generate verifier and syntax check\\
\texttt{-i} & interactive simulation\\
\texttt{-I} & display Promela program after preprocessing\\
\texttt{-nN} & seed for random simulation\\
\texttt{-t} & guided simulation with trail\\
\texttt{-tN} & guided simulation with Nth trail\\
\texttt{-uN} & maximum number of steps is N\\
&\\
\texttt{-f} & translate an LTL formula into a never claim\\
\texttt{-F} & translate an LTL formula in a file into a never claim\\
\texttt{-N} & include never claim from a file\\
&\\
\texttt{-l} & display local variables\\
\texttt{-g} & display global variables\\
\texttt{-p} & display statements\\
\texttt{-r} & display receive events\\
\texttt{-s} & display send events
\end{tabular}

\sct{Compile arguments}
\begin{tabular}{ll}
\texttt{-DBFS} & breadth-first search\\
\texttt{-DNP} & enable detection of non-progress cycles\\
\texttt{-DSAFETY} & optimize for safety\\
\\
\texttt{-DBITSTATE} & bitstate hashing\\
\texttt{-DCOLLAPSE} & collapse compression\\
\texttt{-DHC} & hash-compact compression\\
\texttt{-DMA=n} & minimized DFA with maximum n bytes\\
\texttt{-DMEMLIM=N} & use up to N megabytes of memory
\end{tabular}

\sct{Pan arguments}
\begin{tabular}{ll}
\texttt{-a} & find acceptance cycles\\
\texttt{-f} & weak fairness\\
\texttt{-l} & find non-progress cycles
\end{tabular}

\newpage

\begin{tabular}{ll}
\texttt{-cN} & stop after Nth error\\
\texttt{-c0} & report all errors\\
\texttt{-e} & create trails for all errors\\
\texttt{-i} & search for shortest path to error\\
\texttt{-I} & approximate search for shortest path to error\\
\texttt{-mN} & maximum search depth is N\\
\texttt{-wN} & $2^{N}$ hash table entries\\
\\
\texttt{-A} & suppress reporting of assertion violations\\
\texttt{-E} & suppress reporting of invalid end states
\end{tabular}

\sct{Caveats}
\vspace*{-2ex}
\begin{itemize}
\item Expessions must be side-effect free.
\item Local variable declarations always take effect at the beginning of a process.
\item A \texttt{true} guard can always be selected; an \texttt{else} guard is selected only if all others are false.
\item Macros and \texttt{inline} do \emph{not} create a new scope.
\item Place labels before an \texttt{if} or \texttt{do}, \emph{not} before a guard.
\item In an \texttt{if} or \texttt{do} statement, interleaving can occur between a guard and the following statement.
\item Processes are activated and die in LIFO order.
\item Atomic propositions in LTL formulas must be identifiers starting with lowerase letters and must be boolean variables or symbols for boolean-valued expressions. 
\item Arrays of \texttt{bit} or \texttt{bool} are stored in bytes.
\item The type of a message field of a channel cannot be an array;
it can be a \texttt{typedef} that contains an array.
\item The functions \texttt{empty} and \texttt{full} cannot be negated.
\end{itemize}

\sct{References}
\vspace*{-2ex}
\begin{itemize}
\item G. J. Holzmann. \textit{The Spin Model Checker: Primer and Reference Manual},
Addison-Wesley, 2004.\\\url{http://spinroot.com}.
\item M. Ben-Ari. \textit{Principles of the Spin Model Checker},
Springer, 2008.\\\url{http://www.springer.com/978-1-84628-769-5}.
\end{itemize}
\end{document}
